\section{Описание}
Целью лабораторных работ по курсу «Информационный поиск» является построение полноценной поисковой системы, ориентированной на тематический корпус документов — «Автомобильные статьи». Работа включает в себя полный цикл подготовки данных и реализации базовых компонентов поисковой инфраструктуры.

В рамках работы решаются следующие задачи:
\begin{itemize}
    \item Сбор и нормализация корпуса из открытых источников (английская Wikipedia и auto.ru).
    \item Анализ структуры HTML-документов: выявление метаинформации, разметки, навигационных и служебных блоков.
    \item Извлечение чистого текстового содержимого с удалением рекламы, шаблонов и несемантических элементов.
    \item Реализация поискового робота (краулера) на языке Python 3 с поддержкой вежливого поведения и идемпотентности (повторная обкачка без дублирования).
    \item Разработка модулей предобработки текста на C++: токенизатор и стеммер, не использующие стандартную библиотеку шаблонов (STL) за исключением \texttt{std::string} и \texttt{std::vector}.
    \item Эмпирическая проверка статистических закономерностей языка — в частности, закона Ципфа.
    \item Построение бинарных прямого и обратного индексов.
    \item Реализация булевого поисковика с поддержкой логических операций \texttt{\&\&}, \texttt{||}, \texttt{!} и скобочной группировки.
\end{itemize}

В качестве источников выбраны:
\begin{itemize}
    \item \textbf{Английская Wikipedia} — содержит энциклопедические статьи о моделях автомобилей, производителях, технологиях, исторических событиях. Контент структурирован, проверен сообществом, легко доступен через HTTP.
    \item \textbf{Auto.ru} — российский портал с обзорами, новостями, техническими характеристиками и аналитикой. Используется для расширения корпуса актуальными данными.
\end{itemize}

Краулер сохраняет документы в в MongoDB. Это упрощает последующую индексацию и делает систему автономной.

В результате сформирован корпус из \textbf{50\,218 документов} со средним размером \textbf{7\,800 символов}. Корпус прошёл полную очистку: удалены блоки \texttt{<div class="navbox">}, таблицы \texttt{<table class="infobox">}, скрипты, стили, сноски и другие элементы, не относящиеся к основному содержанию статьи.

\pagebreak