\section{Выводы}
Выполнив лабораторные работы по курсу \enquote{Информационный поиск}, я получил целостное представление о том, как устроена поисковая система «с нуля» — от сбора данных до обработки запросов. На практике я реализовал все ключевые компоненты, которые в реальных поисковиках масштабируются до миллиардов документов, но в лабораторных условиях остаются полностью контролируемыми и понятными.

Во-первых, я освоил принципы работы поисковых роботов: научился корректно обходить веб-сайты, соблюдая правила вежливости, извлекать семантически значимый контент и удалять шум. Это дало понимание, насколько важна качественная предобработка данных — даже самый продвинутый ранжировщик бесполезен, если индекс построен на «грязных» текстах.

Во-вторых, я углубил знания в области лингвистической обработки текста. Реализация токенизатора и стеммера на C++ без использования сторонних библиотек (в том числе большей части STL) научила меня работать на низком уровне: управлять памятью, оптимизировать циклы и избегать скрытых накладных расходов. Я лучше понял, как устроены классические алгоритмы, такие как стеммер Портера, и почему они до сих пор актуальны.

В-третьих, проверка закона Ципфа не была просто «галочкой» — она показала, что собранный корпус действительно обладает статистическими свойствами естественного языка. Это важный момент: он подтверждает, что дальнейшие эксперименты (например, с TF-IDF или BM25) будут иметь смысл и давать предсказуемые результаты.

Наконец, реализация булева поиска с поддержкой скобок и логических операций продемонстрировала силу простых, но хорошо продуманных алгоритмов. Использование отсортированных списков \texttt{doc\_id} и операций слияния позволило достичь времени отклика менее 1 мс даже на корпусе из 50 тыс. документов — это наглядный пример того, как правильная структура данных может заменить сложную логику.

Полученные навыки имеют прямое практическое применение: они лежат в основе не только веб-поиска, но и систем рекомендаций, анализа логов, внутреннего поиска в корпоративных базах знаний и даже некоторых компонентов ИИ-ассистентов. Особенно ценно, что теперь я понимаю не только «как использовать», но и «как устроено внутри» — а это ключевое отличие инженера от потребителя API.

\pagebreak